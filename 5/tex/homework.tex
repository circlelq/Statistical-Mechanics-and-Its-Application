\documentclass[12pt]{article}
\input{/Users/circle/Documents/博一下/homework/setting.tex}
\setcounter{secnumdepth}{2}
\usepackage{bm}
\usepackage{autobreak}
\usepackage{amsmath}
\setlength{\parindent}{2em}
\graphicspath{{../}}

%pdf文件设置
\hypersetup{
	pdfauthor={袁磊祺},
	pdftitle={统计力学及应用作业5}
}

\title{
		\vspace{-1in} 	
		\usefont{OT1}{bch}{b}{n}
		\normalfont \normalsize \textsc{\LARGE Peking University}\\[1cm] % Name of your university/college \\ [25pt]
		\horrule{0.5pt} \\[0.5cm]
		\huge \bfseries{统计力学及应用作业5} \\
		\horrule{2pt} \\[0.5cm]
}
\author{
		\normalfont 								\normalsize
		College of Engineering \quad 2001111690  \quad 袁磊祺\\	\normalsize
        \today
}
\date{}

\begin{document}

%%%%%%%%%%%%%%%%%%%%%%%%%%%%%%%%%%%%%%%%%%%%%%
\captionsetup[figure]{name={图},labelsep=period}
\captionsetup[table]{name={表},labelsep=period}
\renewcommand\contentsname{目录}
\renewcommand\listfigurename{插图目录}
\renewcommand\listtablename{表格目录}
\renewcommand\refname{参考文献}
\renewcommand\indexname{索引}
\renewcommand\figurename{图}
\renewcommand\tablename{表}
\renewcommand\abstractname{摘\quad 要}
\renewcommand\partname{部分}
\renewcommand\appendixname{附录}
\def\equationautorefname{式}%
\def\footnoteautorefname{脚注}%
\def\itemautorefname{项}%
\def\figureautorefname{图}%
\def\tableautorefname{表}%
\def\partautorefname{篇}%
\def\appendixautorefname{附录}%
\def\chapterautorefname{章}%
\def\sectionautorefname{节}%
\def\subsectionautorefname{小小节}%
\def\subsubsectionautorefname{subsubsection}%
\def\paragraphautorefname{段落}%
\def\subparagraphautorefname{子段落}%
\def\FancyVerbLineautorefname{行}%
\def\theoremautorefname{定理}%
\crefname{figure}{图}{图}
\crefname{equation}{式}{式}
\crefname{table}{表}{表}
%%%%%%%%%%%%%%%%%%%%%%%%%%%%%%%%%%%%%%%%%%%

\maketitle

\section{1}


以$u(q)=\sum \limits_{\substack{i,j\\i\not= j}}\frac{1}{2}k\left(q_i-q_j\right)^2$为例,用相似性方法给出一个$F$的表达式。

根据相似性方法给出的一般表达式
\begin{equation}
	F = -3NT\left(\frac{1}{2}+\frac{1}{n}\right)\ln T - TN \ln f\left(\frac{V}{N}T^{-\frac{3}{n}}\right).
\end{equation}

对于
\begin{equation}
	u(q)=\sum {\substack{i,j\\i\not= j}}\frac{1}{2}k\left(q_i-q_j\right)^2,
\end{equation}
可知$n=2$,所以
\begin{equation}
	F = -3NT\ln T - TN \ln f\left(\frac{V}{N}T^{-\frac{3}{2}}\right).
\end{equation}



\section{2}

以麦克斯韦分布推导理想气体自由能$F$,得到物态方程。

由于
\begin{equation}
	F=-N T \ln \left(\frac{\mathrm{e}}{N} \sum_{k} \mathrm{e}^{-\frac{\varepsilon_{k}}{T}}\right),
	\label{eq:21}
\end{equation}
利用这个公式,就能够把相同粒子所组成的而且遵循玻尔兹曼统计的任何气体 的自由能计算出来.

气体分子的平动是准经典的,并且分子的能量可以写成形式
\begin{equation}
	\varepsilon_{k}\left(p_{x}, p_{y}, p_{z}\right)=\frac{p_{x}^{2}+p_{y}^{2}+p_{z}^{2}}{2 m}+\varepsilon_{k}^{\prime}
\end{equation}
式中第一项是分子的平动动能,而 $\varepsilon_{k}^{\prime}$ 代表与分子的转动和它的内部状态相对应 的能级 $; \varepsilon_{k}^{\prime}$ 既与速度无关,也与分子质心的坐标无关 $($ 假定没有任何外场) $.$ 公式 \cref{eq:21}中对数号下的配分函数,现在应该用下式来代替:
\begin{equation}
	\sum_{k} \iint \exp \left(-\frac{\varepsilon_{k}(\boldsymbol{p})}{T}\right) \frac{\mathrm{d}^{3} p}{(2 \pi \hbar)^{3}} \mathrm{~d} V=V\left(\frac{m T}{2 \pi \hbar^{2}}\right)^{3 / 2} \sum_{k} \mathrm{e}^{-\varepsilon_{k} / T}.
\end{equation}



(对 $\mathrm{d} V=\mathrm{d} x \mathrm{~d} y \mathrm{~d} z$ 的积分是遍及气体的整个体积 $V$ 来进行的 ). 于是我们得到自由能为:
\begin{equation}
	F=-N T \ln \left[\frac{\mathrm{e} V}{N}\left(\frac{m T}{2 \pi \hbar^{2}}\right)^{3 / 2} \sum_{k} \mathrm{e}^{-\varepsilon{'}_{k}/ T}\right].
\end{equation}

当然,如果对分子的性质不作任何假定,出现在上式中的和式是不可能以 普遍形式计算出来的. 但是值得注意的是 :它只是温度的函数.
\begin{equation}
	F=-N T \ln \frac{\mathrm{e} V}{N}+N f(T),
\end{equation}
式中 $f(T)$ 是温度的某个函数. 由此可得气体的压强为
\begin{equation}
	P=-\frac{\partial F}{\partial V}=\frac{N T}{V},
\end{equation}
或
\begin{equation}
	P V=N T.
\end{equation}






% \nocite{*}

\newpage
\bibliographystyle{unsrtnat}

\phantomsection

\addcontentsline{toc}{section}{参考文献} %向目录中添加条目,以章的名义
\bibliography{homework}

\end{document}
