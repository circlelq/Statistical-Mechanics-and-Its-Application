\documentclass[12pt]{article}
\input{/Users/circle/Documents/博一下/homework/setting.tex}
\setcounter{secnumdepth}{2}
\usepackage{autobreak}
\usepackage{amsmath}
\setlength{\parindent}{2em}
\graphicspath{{../}}
\ziju{0.1pt}

%pdf文件设置
\hypersetup{
	pdfauthor={袁磊祺},
	pdftitle={统计力学及应用作业7}
}

\title{
		\vspace{-1in} 	
		\usefont{OT1}{bch}{b}{n}
		\normalfont \normalsize \textsc{\LARGE Peking University}\\[0.2cm] % Name of your university/college \\ [25pt]
		\horrule{0.5pt} \\[0.2cm]
		\huge \bfseries{统计力学及应用作业7} \\[-0.2cm]
		\horrule{2pt} \\[0.2cm]
}
\author{
		\normalfont 								\normalsize
		College of Engineering \quad 2001111690  \quad 袁磊祺\\	\normalsize
        \today
}
\date{}

\begin{document}

\input{/Users/circle/Documents/博一下/homework/setc.tex}

\maketitle

\section{1}

在一个装有理想气体的容器上开一小孔,由麦克斯韦分布求出
\begin{enumerate}
	\item 单位时间小孔中跑出的气体分子数。
	\item 跑出的气体分子的平均能量。
	\item 求$\gamma_{12},\ \gamma_{22}$.
\end{enumerate}

\subsection{1}

假设面元$\dif A$在$x,y$平面内,单位时间$\dif t$矢量为$\bm{v}\sim \bm{v}+\dif \bm{v}$的碰壁分子数为
\begin{equation}
	\dif N'(v_x,v_y,v_z) = nf(v_x)f(v_y)f(v_z)\dif v_x \dif v_y \dif v_z \cdot v_x \dif t \dif A.
\end{equation}
对$v_y,v_z$积分得
\begin{equation}
	\dif N'(v_x) = nf(v_x)\dif v_x \cdot v_x \dif t \dif A.
\end{equation}
因为只有$v_x>0$的部分才能碰壁,所以
\begin{equation}
	\begin{aligned}
		N^{\prime} & =n \int_{0}^{\infty} f\left(v_{x}\right) v_{x} \dif v_{x} \cdot \dif A \dif t                                                                                \\
		           & =n \int_{0}^{\infty}\left(\frac{m}{2 \pi k T}\right)^{1 / 2} \cdot \exp \left(-\frac{m v_{x}^{2}}{2 k T}\right) \cdot v_{x} \dif v_{{x}} \cdot \dif A \dif t \\
		           & =n \sqrt{\frac{k T}{2 \pi m}} \dif A \dif t=\frac{1}{4} n \bar{v} \dif A \dif t.
	\end{aligned}
\end{equation}
其中$\bar{v}=\sqrt{\frac{8k T}{ \pi m}}$为麦克斯韦分布的平均速率。单位时间内捧在单位面积上的总分子数为
\begin{equation}
	\Gamma = \frac{N'}{\dif A \dif t} = \frac{1}{4}n\bar{v}.
\end{equation}
对于理想气体,利用$p=nkT$,则有
\begin{equation}
	\Gamma = \frac{n}{4} \sqrt{\frac{8k T}{ \pi m}}=\frac{p}{\sqrt{2\pi mkT}}.
\end{equation}
即为单位时间从单位面积小孔中跑出的气体分子数。

\subsection{2}

分子束的速率分布函数正比于$f(v)v$,故
\begin{equation}
	F(v)\dif v = Af(v)v\dif v,
\end{equation}
由归一化条件得$A=\frac{1}{\bar{v}}$.分子束的方均根速率为
\begin{equation}
	{\bar{\left(v^2_{\text{束}}\right)}}=\int^{\infty}_0F(v)v^2\dif v={\frac{4kT}{m}}.
\end{equation}
跑出的气体分子的平均能量为
\begin{equation}
	e = \frac{1}{2} m \bar{\left(v^2_{\text{束}}\right)}/k = 2T.
\end{equation}


\subsection{3}

设小孔面积为$\sigma$,单位时间小孔交换的粒子数为
\begin{equation}
	\Delta N = \frac{1}{4}(n_1-n_2)\bar{v}\sigma\Delta t= \frac{1}{4}\frac{\Delta p}{kT}\bar{v}\sigma\Delta t.
\end{equation}
\begin{equation}
	\Delta E = \Delta N k\cdot 2T = 2T\frac{1}{4}\frac{\Delta p}{T}\bar{v}\sigma \Delta t.
\end{equation}
又根据
\begin{equation}
	\frac{\Delta E}{\Delta t} = \gamma_{12} \frac{1}{T} \frac{T}{p} \Delta p,
\end{equation}
所以
\begin{equation}
	\gamma_{12} = \frac{p\bar{v}\sigma}{2}.
\end{equation}
\begin{equation}
	\gamma_{22} = \frac{\gamma_{12}}{2T} = \frac{p\bar{v}\sigma}{4T}.
\end{equation}





% \nocite{*}

% \input{bib.tex}

\end{document}

