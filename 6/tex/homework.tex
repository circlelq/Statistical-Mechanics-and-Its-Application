\documentclass[12pt]{article}
\input{/Users/circle/Documents/博一下/homework/setting.tex}
\setcounter{secnumdepth}{2}
\usepackage{autobreak}
\usepackage{amsmath}
\setlength{\parindent}{2em}
\graphicspath{{../}}
\ziju{0.1pt}

%pdf文件设置
\hypersetup{
	pdfauthor={袁磊祺},
	pdftitle={统计力学及应用作业6}
}

\title{
		\vspace{-1in} 	
		\usefont{OT1}{bch}{b}{n}
		\normalfont \normalsize \textsc{\LARGE Peking University}\\[0.2cm] % Name of your university/college \\ [25pt]
		\horrule{0.5pt} \\[0.2cm]
		\huge \bfseries{统计力学及应用作业6} \\[-0.2cm]
		\horrule{2pt} \\[0.2cm]
}
\author{
		\normalfont 								\normalsize
		College of Engineering \quad 2001111690  \quad 袁磊祺\\	\normalsize
        \today
}
\date{}

\begin{document}

\input{/Users/circle/Documents/博一下/homework/setc.tex}

\maketitle

\section{1}


以$P$和$S$为自变量,证明$\left\langle \Delta S\Delta P\right\rangle =0,\ \left\langle \left(\Delta S\right) ^2\right\rangle = C_P,\ \left\langle \left(\Delta P\right) ^2\right\rangle = -T \left(\frac{\partial P}{\partial V} \right)_S$.

根据书上P300有
\begin{equation}
	w \propto \exp \left[\frac{1}{2T}\left(\frac{\partial V}{\partial P}\right)_S\left(\Delta P\right)^2 - \frac{1}{2C_p} \left( \Delta S \right) ^2\right],
\end{equation}
这个式子分解成两个因子,各自只与$\Delta P$或$\Delta S$有关。换句话说,压强和熵是的涨落是统计独立的,因而
\begin{align}
	\left\langle \Delta S\Delta P\right\rangle =         & \ 0,                                                                                                                                    \\
	\left\langle \left(\Delta S\right) ^2\right\rangle = & \ -\frac{1}{2} \frac{1}{-\frac{1}{2C_p}} =  C_P,                                                                                        \\
	\left\langle \left(\Delta P\right) ^2\right\rangle = & \ -\frac{1}{2} \frac{1}{\frac{1}{2T \left(\frac{\partial P}{\partial V} \right)_S}} = -T \left(\frac{\partial P}{\partial V} \right)_S.
\end{align}









% \nocite{*}

% \input{bib.tex}

\end{document}

