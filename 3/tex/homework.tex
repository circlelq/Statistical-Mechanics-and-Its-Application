\documentclass[12pt]{article}
\input{/Users/circle/Documents/博一下/homework/setting.tex}
\setcounter{secnumdepth}{2}
\usepackage{bm}
\usepackage{autobreak}
\usepackage{amsmath}
\setlength{\parindent}{2em}
\graphicspath{{../}}

%pdf文件设置
\hypersetup{
	pdfauthor={袁磊祺},
	pdftitle={统计力学及应用作业3}
}

\title{
		\vspace{-1in} 	
		\usefont{OT1}{bch}{b}{n}
		\normalfont \normalsize \textsc{\LARGE Peking University}\\[1cm] % Name of your university/college \\ [25pt]
		\horrule{0.5pt} \\[0.5cm]
		\huge \bfseries{统计力学及应用作业3} \\
		\horrule{2pt} \\[0.5cm]
}
\author{
		\normalfont 								\normalsize
		College of Engineering \quad 2001111690  \quad 袁磊祺\\	\normalsize
        \today
}
\date{}

\begin{document}

%%%%%%%%%%%%%%%%%%%%%%%%%%%%%%%%%%%%%%%%%%%%%%
\captionsetup[figure]{name={图},labelsep=period}
\captionsetup[table]{name={表},labelsep=period}
\renewcommand\contentsname{目录}
\renewcommand\listfigurename{插图目录}
\renewcommand\listtablename{表格目录}
\renewcommand\refname{参考文献}
\renewcommand\indexname{索引}
\renewcommand\figurename{图}
\renewcommand\tablename{表}
\renewcommand\abstractname{摘\quad 要}
\renewcommand\partname{部分}
\renewcommand\appendixname{附录}
\def\equationautorefname{式}%
\def\footnoteautorefname{脚注}%
\def\itemautorefname{项}%
\def\figureautorefname{图}%
\def\tableautorefname{表}%
\def\partautorefname{篇}%
\def\appendixautorefname{附录}%
\def\chapterautorefname{章}%
\def\sectionautorefname{节}%
\def\subsectionautorefname{小小节}%
\def\subsubsectionautorefname{subsubsection}%
\def\paragraphautorefname{段落}%
\def\subparagraphautorefname{子段落}%
\def\FancyVerbLineautorefname{行}%
\def\theoremautorefname{定理}%
\crefname{figure}{图}{图}
\crefname{equation}{式}{式}
\crefname{table}{表}{表}
%%%%%%%%%%%%%%%%%%%%%%%%%%%%%%%%%%%%%%%%%%%

\maketitle

\section{1}

证明Metropolis方法产生的随机数列分布为$\rho(x)$,若$\frac{A_{xx'}}{A_{x'x}}=\frac{\rho (x')}{\rho (x)}$.

经过长时间的发展后,处于某一状态$x$的概率都为一常数。考虑$x,\ x'$两点间转化的概率,不妨设$\rho(x)>\rho(x')$。
\begin{equation}
	T(x\to x') = \omega_{xx'} A_{xx'} = \omega_{xx'} \frac{\rho (x')}{\rho (x)},
\end{equation}
\begin{equation}
	T(x'\to x) = \omega_{x'x} A_{x'x} = \omega_{x'x}.
\end{equation}
由于
\begin{equation}
	T(x\to x')\rho (x) = \omega_{xx'} \frac{\rho (x')}{\rho (x)} \rho (x) = \omega_{xx'}\rho (x') = \omega_{x'x}\rho (x') = T(x'\to x)\rho (x'),
\end{equation}
所以其达到了细致平衡,也就是说系统达到的平衡是$x$数列分布满足$\rho(x)$概率密度的平衡。\qed








% \nocite{*}

\newpage
\bibliographystyle{unsrtnat}

\phantomsection

\addcontentsline{toc}{section}{参考文献} %向目录中添加条目,以章的名义
\bibliography{homework}

\end{document}
